\chapter{Fundation}

\section{High School}

\begin{enumerate}
    \item Theorem: an important proposition
    \item Lemma: a proposition before a theorem, and is needed for its proof
    \item Corollary: a proposition follows directly from a theorem
\end{enumerate}


\begin{theorem}[multinomial formula]
    for $\alpha=(\alpha_1, ... \alpha_m) \in \naturalnumber^m$ and $x=(x_1, ..., x_m) \realnumber^m$, define these operations:
    \begin{equation}
        \begin{aligned}
            \absolutevalue{\alpha} &= \sum_{j=1}^m a_j \\
            \alpha ! &= \prod_{j=1}^m \alpha_j ! \\
            x^\alpha &= \prod_{j=1}^m {x_j}^{\alpha_j}
        \end{aligned}
    \end{equation}
    
    We have
    \begin{equation}
        \left(\sum_{j=1}^m x_j \right)^k = \sum_{\absolutevalue{\alpha = k}} \frac{k!}{\alpha !} x^\alpha
    \end{equation}
\end{theorem}





\begin{theorem} 
    Here are the essential trigonometry functions:    
    
\begin{equation}
    \begin{aligned}
        \sin(\alpha + \beta) &=\sin\alpha\cos\beta + \cos\alpha\sin\beta \\
        \sin(\alpha - \beta) &=\sin\alpha\cos\beta - \cos\alpha\sin\beta \\
        \cos(\alpha + \beta) &=\cos\alpha\cos\beta - \sin\alpha\sin\beta \\
        \cos(\alpha - \beta) &=\cos\alpha\cos\beta + \sin\alpha\sin\beta \\
        \tan(\alpha+\beta) &= \frac{\tanh\alpha + \tan\beta}{1 - \tan\alpha \tan\beta} \\   
    \end{aligned}
\end{equation}


\begin{equation}
    \begin{aligned}
        \sin\alpha\cos\beta &=\frac{1}{2}[\sin(\alpha+\beta) + \sin(\alpha - \beta)] \\
        \cos\alpha\sin\beta &=\frac{1}{2}[\sin(\alpha+\beta) - \sin(\alpha - \beta)] \\
        \cos\alpha\cos\beta &=\frac{1}{2}[\cos(\alpha + \beta) + \cos(\alpha - \beta)] \\
        \sin\alpha\sin\beta &=-\frac{1}{2}[\cos(\alpha + \beta) - \cos(\alpha - \beta)] \\
    \end{aligned}
\end{equation}


\begin{equation}
    \begin{aligned}   
        \sin\alpha + \sin\beta &= 2 \sin\frac{\alpha+\beta}{2}\cos\frac{\alpha-\beta}{2} \\
        \sin\alpha - \sin\beta &= 2 \cos\frac{\alpha+\beta}{2}\sin\frac{\alpha-\beta}{2} \\
        \cos\alpha + \cos\beta &= 2 \cos\frac{\alpha+\beta}{2}\cos\frac{\alpha-\beta}{2} \\
        \cos\alpha - \cos\beta &= 2 \sin\frac{\alpha+\beta}{2}\sin\frac{\alpha-\beta}{2} \\ 
    \end{aligned}
\end{equation}


\begin{equation}
    \begin{aligned}    
        \sin 2\alpha &= 2 \sin\alpha \cos\alpha \\
        \cos 2\alpha &= \cos^2 \alpha - \sin^2 \alpha \\
        &= 2 \cos^2 \alpha - 1 \\
        &= 1 - 2 \sin^2 \alpha \\
        \sin 3\alpha &= 3 \sin\alpha - 4 \sin^3 \alpha \\
        \cos 3\alpha &= -3 \cos\alpha + 4 \cos^3 \alpha \\
    \end{aligned}
\end{equation}

\begin{equation}
    \begin{aligned}       
        \sin \alpha &= \frac{2 \tan \frac{\alpha}{2}}{1 + \tan^2 \frac{\alpha}{2}} \\
        \cos \alpha &= \frac{1 - \tan^2 \frac{\alpha}{2}}{1 + \tan^2 \frac{\alpha}{2}} \\
        \tan \alpha &= \frac{2 \tan \frac{\alpha}{2}}{1 - \tan^2 \frac{\alpha}{2}} \\  
    \end{aligned}
\end{equation}


\begin{equation}
    a \sin\alpha + b \cos\alpha = \sqrt{a^2 + b^2}\sin\left(\alpha + \arcsin\frac{b}{\sqrt{a^2 + b^2}}\right)
\end{equation}

\end{theorem}



\begin{definition}    
The implication $A \Rightarrow B$ is defined as:
\begin{equation}
    A \Rightarrow B := \neg A \vee B
\end{equation}

Here $A$ is the sufficient condition for $B$, and $B$ is the necessary condition for $A$.

Another observation is:
\begin{equation}
    A \Rightarrow B \Leftrightarrow \neg B \Rightarrow \neg A
\end{equation}

$\neg B \Rightarrow \neg A$ is the contrapositive of the statement $A \Rightarrow B$
\end{definition}

\begin{definition}
    A function $f : X \rightarrow Y$ is a rule that for each element of $X$, there is only one element of $Y$.
\end{definition}

\begin{definition}
    The fiber of $f$ at $y$ is $f^{-1}$ that:
    \begin{equation}
        f^{-1}(y) := \set{x \in X; f(x) = y}
    \end{equation}
\end{definition}

\begin{definition}
    A relation on $X$ which is reflective, transitive and symmetric is called an equivalent relation on $X$, and is denoted as $\sim$.
    
    It defines a surjection $X \rightarrow X / \sim$ which is called quotient function.
\end{definition}



\begin{theorem}
    The natural number $\naturalnumber$ are well ordered, that is, each nonempty subset of $\naturalnumber$ has a minimum.
\end{theorem}


\begin{definition}
    A set $X$ is finite if it is empty, or if there are $n \in \naturalnumber^{\times}$ and a bijection from $\set{1, ..., n}$ to $X$.
    
    If a set is not finite, it is infinite.
    
    A set $X$ is countably infinite if there is a bijection from $X$ to $\naturalnumber$. A set is countable if it is countably infinite or finite. Or it is uncountable.
\end{definition}

\begin{theorem}
    There is no surjection from $X$ to $\powerset{X}$. So $\powerset{\naturalnumber}$ is uncountable.
\end{theorem}
\begin{proof}
    Define a function $\varphi: X \rightarrow \powerset{X}$, and consider the set $A = \set{x\in X, x \notin \varphi(x)}$.
\end{proof}


\begin{theorem}
A countable union of countable sets is countable.    
\end{theorem}

\begin{theorem}
The set $\set{0,1}^{\naturalnumber}$ is uncountable.
\end{theorem}
\begin{proof}
    Define a bijection from $\powerset{\naturalnumber}$ to $\set{0,1}^{\naturalnumber}$.
\end{proof}


\section{Define Numbers}

%
% group theory and define numbers
%

\begin{definition}[magma]
    A magma $(M, \cdot)$ has the property that
    \begin{equation}
        a, b \in M \Rightarrow a \odot b \in M
    \end{equation}
\end{definition}

\begin{definition}[semigroup]
    A semigroup is a magma with associativity:
    \begin{equation}
        (a \odot b) \odot c = a \odot (b \odot c)
    \end{equation}
\end{definition}

\begin{definition}[monoid]
    A monoid is a semigroup with identity:
    \begin{equation}
        e \odot a = a
    \end{equation}
\end{definition}

\begin{definition}[group]
    A group is monoid with inverse:
    \begin{equation}
        \forall a, \exists b \Rightarrow a \odot b = e
    \end{equation}
\end{definition}

\begin{definition}[Abelian group]
    An Abelian group is a group with commutativity on $\cdot$:
    \begin{equation}
        a \odot b = b \odot a
    \end{equation}
\end{definition}

\begin{definition}[ring]
    A ring $(R, +, \odot)$ is defined as:
    \begin{itemize}
        \item $+$ is an Abelian group
        \item $\odot$ is a monoid
        \item Distributivity: $a \odot (b + c) = a \odot b + a \odot c$, and $(a+b) \odot c = a \odot c + b \odot c$
    \end{itemize}
    
\end{definition}

\begin{definition}[field]
    A field is a ring that $\odot$ is a group too.
\end{definition}

\begin{definition}[ordered field]
    An ordered field is a field with order $\leq$:
    \begin{itemize}
        \item $(R, \leq)$ is totally ordered
        \item $x < y \Rightarrow x + z < y + z$
        \item $x,y > 0 \Rightarrow xy > 0$
    \end{itemize}
    
    $x \in R$ is positive if $x > 0$.
\end{definition}


\begin{definition}[coset]
    Let $N$ be a subgroup of $G$ and $g \in G$. Then $g \odot N$ is the left coset and $N \odot g$ is the right coset.
    
    The left coset is an equivalent relation so it defines $G/N$ which is the set of left cosets of $G$ modulo $N$.
\end{definition}

\begin{definition}[normal]
    A subgroup $N$ is normal if for all $g \in G$, we have 
    \begin{equation}
        g \odot N = N \odot g
    \end{equation}    
\end{definition}

\begin{definition}[quotient group]
    A normal subgroup $N$ induces a quotient group $(G/N) \times (G/N) \rightarrow G/N$ that 
    \begin{equation}
        (a \odot N, b \odot N) \rightarrow (a \odot b) \odot N
    \end{equation} 
\end{definition}

\begin{definition}[homomorphism]
    Let $G = (G, \odot)$ and $H = (H, \otimes)$ be groups. A function $\varphi: G \rightarrow H$ is a homomorphism if 
    \begin{equation}
        \varphi(g \odot h) = \varphi(g) \otimes \varphi(h)
    \end{equation}
    
    The image of a homomorphism is a subgroup.
\end{definition}

\begin{theorem}
    If $\varphi$ is a homomorphism, then $\varphi(e) = e'$ and $\varphi(g)^{-1} = \varphi(g^{-1})$.
\end{theorem}

\begin{definition}[kernel]
    Let $\varphi$ be a homomorphism. The kernel of $\varphi$, $\kernel{\varphi}$ is defined as
    \begin{equation}
        \kernel{\varphi} = \varphi^{-1}(e')
    \end{equation}
    
    The kernel is a subgroup.
\end{definition}

\begin{theorem}
Let $\varphi$ be a homomorphism and $N = \kernel{\varphi}$, then
\begin{equation}
    g \odot N = \varphi^{-1}\left(\varphi(g)\right)
\end{equation}
\end{theorem}

\begin{theorem}
    Let $N$ be a normal subgroup of $G$. Then the quotient function $p: G \rightarrow G/N$ is a surjective homomorphism, the quotient homomorphism, with $\kernel{p} = N$.
\end{theorem}


\begin{definition}[isomorphism]
    A homomorphism is isomorphism if it is bijective. If it is from $G$ to itself, it is automorphism. Examples are:
    \begin{itemize}
        \item $g: a \odot g \odot a^{-1}$
        \item for surjective homomorphism $\varphi: G \rightarrow H$, $\varphi' : G/\kernel{\varphi} \rightarrow H$
        \item Let $(G, \odot)$ be a group, but $H$ is not. And $\varphi: G \rightarrow H$ is a bijection. Then define a function $\otimes$ that $g' \otimes h' := \varphi^{-1}(g') \odot \varphi^{-1}(h')$. $\otimes$ is the operation on $H$ induced from $\odot$ by $\varphi$
    \end{itemize}
\end{definition}


Natural number $\naturalnumber$ did not support the inverse $-n$. Integer $\integer$ is the smallest ring that contains $(\naturalnumber, +)$. For convenience we have defined the following:
\begin{itemize}
    \item The additive identity is denoted by $0_R$
    \item The multiplicity identity is denoted by $1_R$
    \item The additive inverse of $a$ is $-a$
    \item $0 \cdot a = a \cdot 0 = 0$
\end{itemize}

Integer $\integer$ did not support the $m/n$. Rational number $\rational$ is the smallest field that contains $\integer$. $\rational$ has a limitation too. For example, there is no rational $x$ that $x^2 = 2$. So we need to extend $\rational$ too.

\begin{definition}[order complete]
    Let $X$ be a totally ordered set. It is order complete if every nonempty subset is bounded below.
\end{definition}

\begin{definition}[Dedekind cut property]
    A set $X$ has Dedekind cut proper if for all nonempty subset $A,B \subset X$ such that $a \leq b$ for all $(a,b) \in A \times B$, there is $c \in X$ that $a \leq c \leq b$ for all $(a,b) \in A \times B$.
    
    Dedekind cut property and order complete are equivalent.
\end{definition}

\begin{theorem}
    $\rational$ is not order complete.
\end{theorem}
\begin{proof}
    Define $A:=\set{x \in \rational; x > 0 , x^2 < 2}$, and $B:=\set{x \in \rational; x > 0, x^2 > 2}$. Assume there is $c$ that $a \leq c \leq b$, define $\xi = \frac{2c+2}{c+2}$, we have
    \begin{equation}
        \begin{aligned}
            \xi^2 - 2 &= \frac{2(c^2 -2)}{(c+2)^2} \\
            \xi - c &= - \frac{c^2 -2}{c+2}    
        \end{aligned}        
    \end{equation}
    
    If $c^2 < 2$, we have $\xi^2 < 2$ and $\xi < c$, so $\xi \in A$ and $\xi > c$, contradiction.
\end{proof}

\begin{definition}[Dedekind cut]
    A Dedekind cut is a partition of the rationals $\rational$ into two subsets $A$ and $B$ such that
    \begin{itemize}
        \item $A$ is not empty
        \item $A \neq \rational$ ($B$ is not empty)
        \item If $x,y \in \rational$, $x < y$, and $y \in A$, then $x \in A$ ($A$ is closed downwards)
        \item If $x \in A$, there is $y \in A$ that $y > x$ (there is no greatest number in $A$)
    \end{itemize}
    
    $A$ and $B$ defines each other. $B$ has a minimal number. 
\end{definition}

So the real number $\realnumber$ is an order completion of rational number $\rational$, which is defined by set $A$ in Dedekind cut. It fills the wholes in $\rational$.

\begin{theorem}
    Here are the common properties of real numbers:
    \begin{itemize}
        \item For each $x \in \realnumber$, there is a $n$ that $n > x$.
        \item If $\displaystyle 0 \leq a \leq \frac{1}{n}$ for all $n \in \naturalnumber^{\times}$, then $a = 0$
        \item For any $a >0$, there is $n \in \naturalnumber^{\times}$ that $\displaystyle \frac{1}{n} < a$
        \item For any $a,b\in \realnumber$ that $a < b$, there is $q \in \rational$ that $a < q < b$
        \item For any $a,b\in \realnumber$ that $a < b$, there is $r \in \realnumber \textbackslash \rational $ that $a < r < b$
    \end{itemize}    
\end{theorem}

The real number $\realnumber$ still has limitation. It cannot solve $x^2 = -1$. The smallest extension field $\complexnumber$ to $\realnumber$ is the complex number.


































































































































































































































































