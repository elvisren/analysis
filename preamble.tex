\documentclass[reqno, 12pt]{book}


% do not use it. conflict with mtp2font
%\usepackage{amssymb}


\usepackage{latexsym}


% set section level depth and re-number it from 1
%\renewcommand{\thesection}{\arabic{section}}
%\setcounter{secnumdepth}{3}


% math theorem, lemma, proof
\usepackage{amsthm}
%\theoremstyle{definition}
%\newtheorem{definition}{Definition}
%\newtheorem{theorem}{Theorem}

\usepackage{thmtools}
\declaretheorem{theorem}
\declaretheorem{definition}
\declaretheorem{example}
\declaretheorem{axiom}



% for mathematical integratoin
\usepackage{commath}


% create index
\usepackage{mathtools}
\usepackage{makeidx}
\makeindex


\usepackage{listings}


% no space in itemize
\usepackage{enumitem}
\setenumerate{itemsep=0pt,partopsep=0pt,parsep=\parskip,topsep=2pt}
\setitemize{itemsep=0pt,partopsep=0pt,parsep=\parskip,topsep=2pt}
\setdescription{itemsep=0pt,partopsep=0pt,parsep=\parskip,topsep=2pt}


% do not float the table
\usepackage{float}


% set line space
\usepackage{setspace}

\usepackage[colorlinks,linkcolor=black,anchorcolor=black,citecolor=black]{hyperref}

% set page size
\usepackage{geometry}
%\geometry{a4paper}
\geometry{a4paper,top=2cm,bottom=2cm,left=3cm,right=2cm}



% set programming code highlight
% \usepackage[chapter]{minted}


% setup bibtex
\usepackage{cite}


% include eps file
\usepackage{epsfig}



% user MathTime professional 2. so expensive...
\usepackage{newtxtext}
\usepackage[T1]{fontenc}
\usepackage[subscriptcorrection,slantedGreek,nofontinfo,mtpcal,mtpfrak,mtphrb,zswash]{mtpro2}



\usepackage{adjustbox}


%
%
% define new commands
%
%


\newcommand\mathhilight[1]{\mathop{\bf #1\/}}

\newcommand\theoref[1]{Theorem~\ref{#1}\footnote{Page~\pageref{#1}}}
\newcommand\defiref[1]{Definition~\ref{#1}\footnote{Page~\pageref{#1}}}


%
%%%%%%%%%%%%%%%%%%%%%%%%%%%%%%%%%%%%%%%%%%%%%% set theory %%%%%%%%%%%%%%%%%%%%%%%%%%%%%%%%%%%%%%%
%

% natural number, the N
\newcommand\naturalnumber[0]{\mathbb{N}}

% real set, the R
\newcommand\realset[0]{\mathbb{R}}

% complex set, the C
\newcommand\complexset[0]{\mathbb{C}}

% the power set P(X)
\newcommand\powerset[1]{\mathcal{P}(#1)}

%
%%%%%%%%%%%%%%%%%%%%%%%%%%%%%%%%%%%%%%%%%%%%%% group theory %%%%%%%%%%%%%%%%%%%%%%%%%%%%%%%%%%%%%%%
%

\newcommand\kernel[1]{\displaystyle \mathrm{Ker}~( #1 )}


%
%%%%%%%%%%%%%%%%%%%%%%%%%%%%%%%%%%%%%%%%%%%%%% linear algebra %%%%%%%%%%%%%%%%%%%%%%%%%%%%%%%%%%%%%%%
%


% null space, the N(x)
\newcommand\nullspace[1]{\mathcal{N}(#1)}

% range space, the R(x)
\newcommand\rangespace[1]{\mathcal{R}(#1)}

% absolute value, the |x|
\newcommand\absolutevalue[1]{\abs{#1}}


% absolute value text, the "abs"
\newcommand\absolutevaluetext[1]{\mathrm{abs}~#1}


% determinate, the same as |x|
\newcommand\determinate[1]{\absolutevalue{#1}}

% determinate text, the "det"
\newcommand\determinatetext[1]{\displaystyle \mathrm{det}~\displaystyle #1}

% the coordinate, the [x]
\newcommand\coordinate[1]{\sbr{#1}}


\newcommand\projection[2]{\mathrm{proj}_{#2} #1}
\newcommand\rowvector[1]{\left[ \displaystyle #1 \right]}
\newcommand\vectorsymbol[1]{\overrightarrow{#1}}

% projection, the x^-1
\newcommand\inverse[1]{ {#1}^{-1} }

% dimension text, the "dim x"
\newcommand\dimension[1]{\displaystyle \mathrm{dim}~\left( #1 \right)}

% rank text, the "rank x"
\newcommand\rank[1]{\displaystyle \mathrm{rank}~\left( #1 \right)}


% adjugate matrix, the "adj x"
\newcommand\adjugate[1]{\displaystyle \mathrm{adj}~\left( #1 \right)}



% inner product, the <x,y>
\newcommand\innerproduct[2]{\left\langle \displaystyle #1, #2 \right\rangle}

% trace, the "tr x"
\newcommand\trace[1]{\displaystyle \mathrm{tr}~( #1 )}

% span, the "span x"
\newcommand\spanset[1]{\displaystyle \mathrm{span}~\left( #1 \right)}


% pseudoinverse, the dagger
\newcommand\pseudoinverse[1]{ {#1}^\dagger }



% transpose, the T
\newcommand\transpose[1]{ {#1}^\top }


%\newcommand\adjugate[1]{\displaystyle \mathrm{adj}~\displaystyle #1 }
\newcommand\cofactor[1]{\displaystyle \mathrm{cof}~\displaystyle #1 }

\newcommand\columnvector[1]{\boldsymbol{#1}}
% \newcommand\norm[1]{\displaystyle \left\lVert #1 \right\rVert}

\newcommand\conjugate[1]{\overline{#1}}

%
%%%%%%%%%%%%%%%%%%%%%%%%%%%%%%%%%%%%%%%%%%%%%% machine learning %%%%%%%%%%%%%%%%%%%%%%%%%%%%%%%%%%%%%%%
%

\newcommand\subscription[2]{\boldsymbol{#1}^{(#2)}}

% argmin and argmax
\newcommand\argmin[1]{\mathop{\arg\min}\limits_{#1}}
\newcommand\argmax[1]{\mathop{\arg\max}\limits_{#1}}

