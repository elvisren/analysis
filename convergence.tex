\chapter{Convergence}


%
%
% convergence of sequence
%
%

\section{Convergence of Sequence}


\begin{definition}[metric]
    Let $X$ be a set. A metric $d: X \times X \rightarrow \positiverealnumber$ has the following property:
    \begin{itemize}
        \item $d(x,y) = 0 \Leftrightarrow x = y$
        \item $d(x,y) = d(y,x)$
        \item $d(x,y) \leq d(x,z) + d(z y)$
    \end{itemize}
\end{definition}

\begin{definition}
    An open ball is
    \begin{equation}
        \openball{x}{r} := \set{y \in X; d(x,y) < r}
    \end{equation}
    
    An closed ball is
    \begin{equation}
        \closedball{x}{r} := \set{y \in X; d(x,y) \leq r}
    \end{equation}
\end{definition}

\begin{definition}[neighborhood]
    A subset $U \subseteq X$ is a neighborhood of $x$ if there is $r>0$ that $\openball{x}{r} \subseteq U$. 
\end{definition}

\begin{definition}[$\epsilon$-neighborhood]
    $\openball{x}{\epsilon}$ and $\closedball{x}{\epsilon}$ are called open and closed $\epsilon$-neighborhood of $x$.
\end{definition}

\begin{definition}[sequence]
    A sequence is a function $\naturalnumber \rightarrow X$, which is written as
    \begin{equation}
        (x_n)
    \end{equation}
\end{definition}

\begin{definition}[cluster point]
    $a$ is a cluster point of $(x_n)$ if every neighborhood of $a$ contains infinitely many terms of the sequence.
    
    For example, $(-1)^n$ has two cluster points: $1$ and $-1$.
\end{definition}

\begin{definition}[limit]
    A sequence $(x_n)$ converges to limit $a$ if every neighborhood of $a$ contains almost all terms of the sequence, which is written as
    \begin{equation}
        \lim_{n \rightarrow \infty} x_n = a \text{    or    } x_n \rightarrow a
    \end{equation}
\end{definition}

\begin{theorem}
    $x_n \rightarrow a$ means for each $\epsilon>0$, there is $m$ that for all $n > m$, $x_n \in \openball{a}{\epsilon}$.
\end{theorem}

\begin{definition}[bounded]
    A subset $Y \subseteq X$ is bounded if there is $M$ that $d(x,y) \leq M$ for all $x,y \in Y$. In this case, the diameter of $Y$ is defined as
    \begin{equation}
        \text{diam }(y) := \underset{x,y \in Y}{\text{sup}} d(x,y)
    \end{equation}
    
    It is easy to show that every convergent sequence is bounded.
\end{definition}

\begin{theorem}
    Let $x_n \rightarrow a$. Then $a$ is the unique cluster point of $(x_n)$. So the limit is unique.
\end{theorem}





%
%
% real and complex and normed vector sequence
%
%

\section{Real and Complex and Normed Vector Sequence}

\begin{theorem}
    Let $(x_n)$ and $(y_n)$ be convergent sequence with limit $a$ and $b$, we have
    \begin{itemize}
        \item $(x_n + y_n)$ converge to $a + b$
        \item $c (x_n)$ converge to $ca$
        \item $(x_n y_n)$ converge to $ab$
    \end{itemize}
\end{theorem}

\begin{theorem}
    For three real sequence $(x_n)$, $(y_n)$ and $(z_n)$ with property that $x_i \leq y_i \leq z_i$ for almost all $i$. If $x_n \rightarrow a$ and $z_n \rightarrow a$, we have $y_n \rightarrow a$.
\end{theorem}

\begin{theorem}
    For a sequence $(z_n)$ in $\complexnumber$, the following are equivalent:
    \begin{itemize}
        \item $(z_n)$ converges
        \item Real and imaginary part converge
    \end{itemize}
\end{theorem}

\begin{definition}
    For a inner product vector space $E$, define a norm as 
    \begin{equation}
        d(x,y) := \vectornorm{x - y}
    \end{equation}
\end{definition}




%
%
% monotone sequence
%
%

\section{Monotone Sequence}

\begin{definition}[increasing]
    A sequence $(x_n)$ is increasing if $x_n \leq x_{n+1}$.
\end{definition}

\begin{theorem}
    Every increasing (decreasing) bounded sequence $(x_n)$ in $\realnumber$ converges. 
\end{theorem}
\begin{proof}
    Since $\realnumber$ is order complete and $(x_n)$ is bounded above, it has upper bound which is the limit.
\end{proof}

Some important limits are:
\begin{itemize}
    \item $\displaystyle \lim_{n \rightarrow \infty} \frac{n^k}{a^n} = 0$
    \item $\displaystyle \lim_{n \rightarrow \infty} \frac{a^n}{n!} = 0$
    \item $\displaystyle \lim_{n \rightarrow \infty} \sqrt[n]{n} = 1$
    \item $\displaystyle \lim_{n \rightarrow \infty} \left(1+\frac{1}{n} \right)^n = e$
\end{itemize}


\section{Infinity Limit}

Extend $\realnumber$ to $\extendedrealnumber$ by adding $\pm \infty$ to the cluster points.

\begin{theorem}
    Every monotone sequence in $\realnumber$ converged in $\extendedrealnumber$.
\end{theorem}


\begin{definition}
    Let $(x_n)$ be a sequence in $\realnumber$. Define limit superior as
    \begin{equation}
        S = \limsup_{n \rightarrow \infty} x_n = \varlimsup_{n\rightarrow \infty} x_n = \lim_{n \rightarrow \infty} \underset{k \geq n}{\text{ sup }} x_k
    \end{equation}
    
    and limit inferior as
    \begin{equation}
        I = \liminf_{n \rightarrow \infty} x_n = \varliminf_{n\rightarrow \infty} x_n = \lim_{n \rightarrow \infty} \underset{k \geq n}{\text{ inf }} x_k
    \end{equation}    
\end{definition}

Be noted that $\displaystyle \varlimsup_{n\rightarrow \infty} x_n$ is a decreasing sequence and $\displaystyle \varliminf_{n\rightarrow \infty} x_n$ is an increasing sequence.


If limit superior S is a real number $s$, it means that for any $\epsilon>0$, there is a $N$ that $s + \epsilon$ is an upper bound for $x_{n > N}$. The reverse is true for limit inferior I.

$[I,S]$ may not contain any number from $(x_n)$. But for any $\epsilon >0$, $[I - \epsilon, S + \epsilon]$ contains all but finite numbers from $(x_n)$, and this is the smallest closed interval with this property.

In general we have 
\begin{equation}
    \inf_{n \rightarrow \infty} x_n \leq \liminf_{n \rightarrow \infty} x_n \leq \limsup_{n \rightarrow \infty} x_n \leq \sup_{n \rightarrow \infty} x_n
\end{equation}

The additivity of limit superior and inferior is
\begin{equation}
    \begin{aligned}
        \liminf_{n \rightarrow \infty} (a_n + b_n) &\geq \liminf_{n \rightarrow \infty} a_n + \liminf_{n \rightarrow \infty} b_n \\
        \limsup_{n \rightarrow \infty} (a_n + b_n) &\leq \limsup_{n \rightarrow \infty} a_n + \limsup_{n \rightarrow \infty} b_n \\
    \end{aligned}
\end{equation}


\begin{theorem}
    A sequence converges in $\realnumber$ when 
    \begin{equation}
        \varlimsup_{n\rightarrow \infty} x_n = \varliminf_{n\rightarrow \infty} x_n \in \realnumber
    \end{equation}
\end{theorem}

\begin{theorem}
    Any sequence $(x_n)$ in $\realnumber$ has a smallest cluster point $x_*$ and greatest cluster point $x^*$ in $\extendedrealnumber$ that
    \begin{equation}
        \begin{aligned}
            \limsup_{n \rightarrow \infty} x_n &= \varlimsup_{n\rightarrow \infty} x_n &= x^* \\
            \liminf_{n \rightarrow \infty} x_n &= \varliminf_{n\rightarrow \infty} x_n &= x_*
        \end{aligned}
    \end{equation}
\end{theorem}
\begin{proof}
    Let's check $x^*$. 
    
    If $x^* = - \infty$. Then for any $m < 0$, there is $n$ that $m > \underset{k \geq n}{\text{ sup }} x_k$, so $x_k < m$ for all $m$. So $x^* = -\infty$ is the only cluster point of $(x_n)$.
    
    If $x^* = \infty$, the case is the same.
        
    Now assume $x^* \in \realnumber$. Since $\underset{k \geq n}{\text{ sup }} x_k \geq x_n$, if $x^*$ is a cluster point, it will be greater than any other cluster point. Now the question becomes whether $x^*$ is a cluster point. For any $\epsilon>0$, there are infinitely many $x_i$ that $x_i > x^* - \epsilon$. For if there are only finite number of $x_i$ that $x_i > x^* - \epsilon$, we have $\underset{k \geq \max i }{\text{ sup }} x_k < x^* - \epsilon$. It is easy to prove that there are infinite $x_i$ that $x_i < x^* + \epsilon$. So $(x^* - \epsilon, x^* + \epsilon)$ contains infinite number of $(x_n)$ and is a cluster point.
\end{proof}

\begin{theorem}[Bolzano-Weierstrass theorem]
    Every bounded sequence in $\realnumber^n$ has a convergent subsequence. 
\end{theorem}









































































































































































































































































































































































































































































































































































































































































































































































